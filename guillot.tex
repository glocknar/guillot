\documentclass{article}

\usepackage{amsmath}
\usepackage{amsfonts}

\usepackage[margin=1in]{geometry}

\begin{document}

\noindent {\large \textbf{PROBLEM 1.2}} \indent (Guillot, 2018, p. 21) \\\\
\indent Let $F$ be a field of characteristic $p \neq 2$. \\\\
\indent Let $a \in F^\times{\setminus}F^{\times2}$, and $K = F[\sqrt{a}]$. \\\\
\\
\textbf{PART 1} \\\\
\indent Suppose there exists an extension $L/F$ with $K \subset L$. \\\\
\indent Suppose that $L/F$ is cyclic with $\textbf{Gal}(L/F) \cong C_4$. \\\\
\indent Show that there exists an $\alpha \in K$ such that $N_{K/F}(\alpha) = -1$. \\\\
\indent \emph{Hint}: try $\alpha = \dfrac{\theta - \sigma^2(\theta)}{\sigma(\theta) - \sigma^3(\theta)}$. \\\\
\\
\textbf{SOLUTION 1.2.1} \\\\
Taking Guillot's hint: let $\alpha$ be of the form $\dfrac{\theta - \sigma^2(\theta)}{\sigma(\theta) - \sigma^3(\theta)}$. \\\\\\
We take $\theta$ to be an element of $L\setminus K$. Such a $\theta$ must exist by dimension. \\\\
We take $\sigma$ to be a generator of $\textbf{Gal}(L/F) \cong C_4$. Let $1_G$ be the identity. \\\\
Since $\sigma$ fixes $L$, $\alpha$ lies in $L$. Since $\sigma$ acts linearly, we have: \\\\\\
\indent $\sigma(\alpha) = \dfrac{\sigma(\theta) - \sigma^3(\theta)}{\sigma^2(\theta) - \theta} = -1/\alpha$. \indent $(1)$ \\\\\\
We seek $N_{L/K}(\alpha)$, which involves only those  $\sigma^n$ that lie in $\textbf{Gal}(L/K)$ \\\\
To find $\textbf{Gal}(L/K)$ we apply \emph{The Fundamental Theorem of Galois Theory} : \\\\
\indent There exists normal series $\{1_G\} \, \triangleleft \, \textbf{Gal}(L/K) \, \triangleleft \, \textbf{Gal}(L/F)$. \\\\
\indent So $\textbf{Gal}(L/K)$ is a normal subgroup in $\textbf{Gal}(L/F)$. \\\\
\indent The only such subgroup in $C_4$ is given by $\{1_G, \, \sigma^2\}$. \\\\
\indent See \textbf{Theorem 5.1} in (Morandi, 1996, p. 51) for details. \\\\
So $N_{L/K}(\alpha) = \alpha \,\, \sigma^2(\alpha)$, and this is equal to $\alpha^2$ by $(1)$ above. \\\\
Since the image of $N_{L/K}$ is $K$, we now know that $\alpha^2 \in K$. \\\\
This implies that $\alpha \in K$, and we may now ask for $N_{K/F}(\alpha)$. \\\\
We have $\textbf{Gal}(K/F) \cong C_2$. Let $\tau$ be the non-identity which conjugates $\sqrt{a}$. \\\\
This automorphism is the same as the restriction of $\sigma$ to $K$, denoted by $\sigma|_K$. \\\\
\indent Such a compatible restriction exists by \emph{The Isomorphism Extension Theorem}. \\\\
\indent See \textbf{Theorem 3.20} in (Morandi, 1996, p. 34) for details. \\\\
\indent To see that $\tau$ is given by $\sigma|_K$, observe that $\dfrac{\textbf{Gal}(L/F)}{\textbf{Gal}(L/K)} \cong \textbf{Gal}(K/F)$. \\\\
\indent Via this homomorphism, the preimage of $\tau$ is the equivalence class of $\sigma$. \\\\
\indent  See \textbf{Theorem 4} in (Pinter, 1990, p. 330). \\\\\\
Therefore, the norm of $\alpha$ with respect to $K/F$ is $\alpha \,\, \sigma|_K(\alpha)$. \\\\
By $(1)$ above, $\alpha \,\, \sigma|_K(\alpha) = N_{K/F}(\alpha) = -1$. $\square$ \\\\\\

\noindent \textbf{PART 2} \\\\
\indent Let $\alpha \in K$ be an element with norm $N_{K/F}(\alpha) = -1$. \\\\
\indent Show that there exists a cyclic extension $L/F$ of degree 4. \\\\
\indent \emph{Hint}: The case when $\alpha \in F$ must be treated separately. \\\\
\indent When $\alpha \notin F$, try $L = K[\sqrt{1+\alpha^2}]$. \\\\
\\
\textbf{SOLUTION 1.2.2} \\\\
\noindent The element $\alpha$ is of the form $x + \sqrt{a} y$ for some $x, y \in F$. First, suppose $\alpha \in F$: \\\\
We have $y = 0$. Since $N_{K/F}(\alpha) = -1$, we also have $x^2 = -1$. \\\\
This implies that $i = \sqrt{-1} \in F$. So $F$ contains a primitive $4^{th}$ root of unity: \\\\
Therefore we may apply $\emph{The Fundamental Theorem of Kummer Theory}$: \\\\ 
\indent Subgroups of order $n \geq 1$ in $F^\times/F^{\times n}$ give rise to Galois extensions of degree $n$. \\\\
\indent Let $[b]$ be such a subgroup. Let $L/F$ be such an extension. \\\\
\indent Then $L$ is generated as $F[\sqrt[n]{b}]$ for any $b$ in $[b]$. \\\\
\indent See \textbf{Theorem 1.25} in (Guillot, 2018, p. 14) for details. \\\\
It suffices to show that such a subgroup and generating element exist for $n = 4$. \\\\
Before we do so, observe that the exponent of $F^\times/F^{\times4}$ is $4$. \indent $(2)$ \\\\
\indent Every element of $[b]$ is of the form $bf^{4}$ for some $b \in F^\times{\setminus}F^{\times4}$ and $f \in F^\times$. \\\\
\indent Furthermore, $bf^4$ is in lowest terms, so that $b$ does not contain a $4^{th}$ power. \\\\
\indent Let $m$ be the order of $[b]$ in $F^\times/F^{\times4}$.  If $m > 4$,  there exists some $c = (bf^{4})^m \in F^{\times 4}$. \\\\
\indent This implies that $b^m = b^{m-4}b^4 = 1$, contradicting that $c$ is given in lowest terms. \\\\
By group theory, the order of any subgroup $[b]$ in $F^\times/F^{\times4}$ must divide the exponent $4$. \\\\
Now it is clear that we may choose $b = a$ and $[b] = [a]$: \\\\
Since $a$ is not a square or a $4^{th}$ power in $F^\times$, the order of $[a]$ cannot be $2$ or $1$. $(3)$ \\\\
So the order of $[a]$ is $4$, and a cyclic extension $L/K$ of degree $4$ exists. \\\\
\\
Now suppose $\alpha \notin F$. We take Guillot's hint and try $L = K[\sqrt{1+\alpha^2}]$. \\\\
Clearly, $K/F$ is Galois of degree $2$ with $\textbf{Gal}(K/F) \cong C_2$. \\\\
Let $\sigma$ be the non-identity (conjugation) and $\overline{\alpha} = \sigma(\alpha)$. \\\\
We have: $N_{K/F}(1 + \alpha^2) = (1 + \alpha^2) \, (1 + \sigma(\alpha)^2) = (1 + \alpha^2) \, (1 + \overline{\alpha}^2).$ \\\\
Recall that $N_{K/F}(\alpha) = \alpha \, \sigma(\alpha) = -1$, and so $\alpha^2 \, \sigma(\alpha)^2 = 1$. \\\\
Applying these ``formulae", we get $N_{K/F}(1 + \alpha^2) = (\alpha - \overline{\alpha})^2$. \\\\
If $1 + \alpha^2$ lies in $K^{\times2}$, then $N_{K/F}(1 + \alpha^2) = N_{K/F}(\beta)^2$ for some $\beta \in K$. \\\\
But $N_{K/F}(\beta) \in F$, which implies that $N_{K/F}(1 + \alpha^2)$ also lies in $F^{\times2}$. \\\\
This cannot be the case since $(\alpha - \sigma(\alpha))^2 = 4y^2a$  does not lie in $F^{\times2}$. \\\\
\indent While $4y^2$ is a square, recall that we have assumed $a \notin F^{\times2}$. \\\\
Therefore, $1 + \alpha^2$ is not a square in $K^{\times}$. \\\\
We may now apply the so-called \emph{Equivariant Kummer Theory}: \\\\
\indent Clearly, $L/K$ is Galois of degree $2$ with $\textbf{Gal}(L/K) \cong C_2$. \\\\
\indent Suppose that $K$ contains a primitive $n^{th}$ root of unity. \\\\
\indent Consider subgroups of order $n \geq 1$ in $K^\times/K^{\times n}$ fixed by the action of $\textbf{Gal}(K/F)$. \\\\
\indent These give rise to Galois extensions $L/K$ of degree $n$ such that $L/F$ is also Galois. \\\\
\indent Let $[b]$ be such a subgroup of $K^\times/K^{\times n}$. Then $L$ is generated as $K[\sqrt[n]{b}]$ for any $b$ in $[b]$.  \\\\
\indent See \textbf{Theorem 1.26} in (Guillot, 2018, p. 14) for details. \\\\
Certainly, $K$ contains the primitive \emph{square} root of unity, namely $-1$. \\\\
Using $(2)$ and $(3)$ above, the order of $[1 + \alpha^2]$ in $K^\times/K^{\times2}$ is $2$. \\\\
It remains to show that $[1 + \alpha^2]$ is preserved by the action of $\textbf{Gal}(K/F)$ : \\\\\\
\indent $\sigma(1 + \alpha^2) = 1 + \overline{\alpha}^2 = \dfrac{N_{K/F}(1 + \alpha^2)}{1 + \alpha^2} = \dfrac{4y^2\alpha}{1 + \alpha^2} = (1 + \alpha^2) \cdot \left(\dfrac{2y\sqrt{\alpha}}{1 + \alpha^2}\right)^2$. \\\\\\
Thus, $\sigma(1 + \alpha^2) = (1 + \alpha^2) \cdot k^2$ for $k = \left(\dfrac{2y\sqrt{\alpha}}{1 + \alpha^2}\right)^2 \in K^\times$, and $\sigma$ fixes $[1 + \alpha^2]$. $\square$ \\\\
\\

\noindent \textbf{PART 3} \\\\
\indent Show that the following are equivalent: \\\\
\indent \indent (a) \indent There exists an $\alpha \in K$ with $N_{K/F}(\alpha) = -1$, and \\\\
\indent \indent (b) \indent $a$ is a sum of two squares in $F$. \\\\
\\
\textbf{SOLUTION 1.2.3} \\\\
\noindent Recall that $F$ is a field of characteristic $p \neq 2$ and $K = F[\sqrt{a}]$. \\\\
Clearly, $K/F$ is Galois of degree $2$ with $\textbf{Gal}(K/F) \cong C_2$. \\\\
Let $\sigma$ be the non-identity (conjugation) which sends $\sqrt{a}$ to $-\sqrt{a}$. \\\\
If $a = x^2 + y^2$ for some $y \neq 0$ and $x$ in $F$, then $\dfrac{x^2-a}{y^2} = -1$. Also, \\\\\\
\indent $\dfrac{x + \sqrt{a}}{y} \cdot \dfrac{x - \sqrt{a}}{y} =\dfrac{x + \sqrt{a}}{y} \cdot \sigma(\dfrac{x + \sqrt{a}}{y}) = -1$. \\\\\\
Letting $\alpha = \dfrac{x + \sqrt{a}}{y} \in K$, the equation above reduces to $N_{K/F}(\alpha) = -1$. \\\\
On the other hand, if $y = 0$, then $a = x^2$. This cannot be the case because $a \notin F^{\times2}$. \\\\
Therefore (b) $\implies$ (a). \\\\\\
Every element of $K$ is of the form $x + \sqrt{a} y$ for some $x, y \in F$. \\\\
If $N_{K/F}(\alpha) = -1$, then $(x + \sqrt{a}y) \cdot (x - \sqrt{a}y) = x^2 - ay^2 = -1$. \\\\
If $y = 0$, then $a = x^2$. This cannot be the case because $a \notin F^{\times2}$. \\\\
So we can safely rearrange $x^2 - ay^2 = -1$ as $a = \left(\dfrac{1}{y}\right)^2 + \left(\dfrac{x}{y}\right)^2$. \\\\\\
Therefore (a) $\implies$ (b). $\square$
\pagebreak

\noindent {\Large \textbf{REFERENCES}} \\\\\\
Guillot, P. (2018). \emph{A Gentle Course in Local Class Field Theory: Local Number Fields, Brauer Groups, Galois Cohomology}. Cambridge: Cambridge University Press. \\\\
Morandi, P. (1996). \emph{Field and Galois Theory}. Graduate Texts in Mathematics, vol 167. Springer, New York, NY. \\\\
Pinter, C.C. (1990) \emph{A Book of Abstract Algebra}. 2nd Edition, Dover Publications, Inc., Mineola, New York. \\\\

\end{document}