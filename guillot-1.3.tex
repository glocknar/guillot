\documentclass{article}

\usepackage{amsmath}
\usepackage{amsfonts}

\usepackage[margin=1in]{geometry}

\begin{document}

\noindent {\large \textbf{PROBLEM 1.3}} \indent (Guillot, 2018, p. 21) \\\\
\indent Let $F$ be a field of characteristic $p \neq 2$. \\\\
\indent Suppose the equation $x^2 - y^2 = 1$ has a solution with $x, y \in F$. \\\\
\indent Let $H$ be a subgroup of $D_8$, the dihedral group of order 8. \\\\
\indent Show that there exists a Galois extension $L/F$ such that: \\\\
\indent $\bullet$ \indent $F[\sqrt{a}, \sqrt{b}] \subset L$, and \\\\
\indent $\bullet$ \indent $\textbf{Gal}(L/F) \cong H$. \\\\
\\
\textbf{SOLUTION 1.2.1} \\\\
This problem is a generalization of \textbf{Proposition 1.28} in (Guillot, 2018, p. 16). \\\\
Here we do not assume that $[a]$ and $[b]$ are linearly independent classes in $F^\times/F^{\times2}$. \\\\
As a result $\textbf{Gal}(L/F) \cong H$ rather than $\textbf{Gal}(L/F) \cong D_8$. \\\\
As Guillot hints, there are many possibilities to consider here. \\\\
These arise in part due to the different ways in which $[a]$ and $[b]$ can be dependent: \\\\
\indent First, note that $F^\times/F^{\times2}$ is a vector space over $\mathbb{Z}/2\mathbb{Z}$. \\\\
\indent The scalars are either $0$ or $1$. The vectors are classes such as $[a]$ and $[b]$. \\\\
\indent These vectors form an abelian group under \emph{multiplication} (not addition). \\\\
\indent Accordingly, the scalar operation is really \emph{exponentiation} (not multiplication). \\\\
\indent For example, consider $\mathbb{Q^\times}/\mathbb{Q}^{\times2}$, which is also a vector space over $\mathbb{Z}/2\mathbb{Z}$. \\\\
\indent The scalars are again 0 or 1. The vectors are classes $[p]$ for prime $p$ and $[-1]$. \\\\
\indent Indeed, every element of $\mathbb{Q^\times}/\mathbb{Q}^{\times2}$ can be written as $\pm p_1^{n_1} p_2^{n_2} p_3^{n_3}...$ where the $n_k$ are $0$ or $1$. \\\\
\indent Thus, we interpret the dependence of $[a]$ and $[b]$ as follows: \\\\
\indent \indent There exist scalars $c, d \in \mathbb{Z}/2\mathbb{Z}$ not both $0$ such that $[a]^c = [b]^d$. \\\\
\indent So $[a]$ and $[b]$ can be linearly dependent if either equals $[1]$ or if $[a] = [b]$. \\\\
\indent This is an application of \textbf{Example 1.4} in (Guillot, 2018, pp. 4-5). \\\\
\\
\noindent \textbf{PART 1:} The case when $[a] = [1] = [b]$. \\\\
\\
\noindent \textbf{PART 2:} The case when $[a] \neq [1] = [b] $. \\\\
\\
\noindent \textbf{PART 3:} The case when $[a] = [1] \neq 1$. \\\\
\\
\noindent \textbf{PART 4:} The case when $[a] = [b] \neq 1$. \\\\
\\
\noindent {\Large \textbf{REFERENCES}} \\\\\\
Guillot, P. (2018). \emph{A Gentle Course in Local Class Field Theory: Local Number Fields, Brauer Groups, Galois Cohomology}. Cambridge: Cambridge University Press. \\\\
Morandi, P. (1996). \emph{Field and Galois Theory}. Graduate Texts in Mathematics, vol 167. Springer, New York, NY. \\\\
Pinter, C.C. (1990) \emph{A Book of Abstract Algebra}. 2nd Edition, Dover Publications, Inc., Mineola, New York. \\\\

\end{document}